% Options for packages loaded elsewhere
\PassOptionsToPackage{unicode}{hyperref}
\PassOptionsToPackage{hyphens}{url}
%
\documentclass[
]{article}
\title{A basic workflow for using the GMACS repository for developpers}
\author{Matthieu VERON}
\date{Last compiled on 13 June, 2022}

\usepackage{amsmath,amssymb}
\usepackage{lmodern}
\usepackage{iftex}
\ifPDFTeX
  \usepackage[T1]{fontenc}
  \usepackage[utf8]{inputenc}
  \usepackage{textcomp} % provide euro and other symbols
\else % if luatex or xetex
  \usepackage{unicode-math}
  \defaultfontfeatures{Scale=MatchLowercase}
  \defaultfontfeatures[\rmfamily]{Ligatures=TeX,Scale=1}
\fi
% Use upquote if available, for straight quotes in verbatim environments
\IfFileExists{upquote.sty}{\usepackage{upquote}}{}
\IfFileExists{microtype.sty}{% use microtype if available
  \usepackage[]{microtype}
  \UseMicrotypeSet[protrusion]{basicmath} % disable protrusion for tt fonts
}{}
\makeatletter
\@ifundefined{KOMAClassName}{% if non-KOMA class
  \IfFileExists{parskip.sty}{%
    \usepackage{parskip}
  }{% else
    \setlength{\parindent}{0pt}
    \setlength{\parskip}{6pt plus 2pt minus 1pt}}
}{% if KOMA class
  \KOMAoptions{parskip=half}}
\makeatother
\usepackage{xcolor}
\IfFileExists{xurl.sty}{\usepackage{xurl}}{} % add URL line breaks if available
\IfFileExists{bookmark.sty}{\usepackage{bookmark}}{\usepackage{hyperref}}
\hypersetup{
  pdftitle={A basic workflow for using the GMACS repository for developpers},
  pdfauthor={Matthieu VERON},
  hidelinks,
  pdfcreator={LaTeX via pandoc}}
\urlstyle{same} % disable monospaced font for URLs
\usepackage[margin=1in]{geometry}
\usepackage{color}
\usepackage{fancyvrb}
\newcommand{\VerbBar}{|}
\newcommand{\VERB}{\Verb[commandchars=\\\{\}]}
\DefineVerbatimEnvironment{Highlighting}{Verbatim}{commandchars=\\\{\}}
% Add ',fontsize=\small' for more characters per line
\usepackage{framed}
\definecolor{shadecolor}{RGB}{248,248,248}
\newenvironment{Shaded}{\begin{snugshade}}{\end{snugshade}}
\newcommand{\AlertTok}[1]{\textcolor[rgb]{0.94,0.16,0.16}{#1}}
\newcommand{\AnnotationTok}[1]{\textcolor[rgb]{0.56,0.35,0.01}{\textbf{\textit{#1}}}}
\newcommand{\AttributeTok}[1]{\textcolor[rgb]{0.77,0.63,0.00}{#1}}
\newcommand{\BaseNTok}[1]{\textcolor[rgb]{0.00,0.00,0.81}{#1}}
\newcommand{\BuiltInTok}[1]{#1}
\newcommand{\CharTok}[1]{\textcolor[rgb]{0.31,0.60,0.02}{#1}}
\newcommand{\CommentTok}[1]{\textcolor[rgb]{0.56,0.35,0.01}{\textit{#1}}}
\newcommand{\CommentVarTok}[1]{\textcolor[rgb]{0.56,0.35,0.01}{\textbf{\textit{#1}}}}
\newcommand{\ConstantTok}[1]{\textcolor[rgb]{0.00,0.00,0.00}{#1}}
\newcommand{\ControlFlowTok}[1]{\textcolor[rgb]{0.13,0.29,0.53}{\textbf{#1}}}
\newcommand{\DataTypeTok}[1]{\textcolor[rgb]{0.13,0.29,0.53}{#1}}
\newcommand{\DecValTok}[1]{\textcolor[rgb]{0.00,0.00,0.81}{#1}}
\newcommand{\DocumentationTok}[1]{\textcolor[rgb]{0.56,0.35,0.01}{\textbf{\textit{#1}}}}
\newcommand{\ErrorTok}[1]{\textcolor[rgb]{0.64,0.00,0.00}{\textbf{#1}}}
\newcommand{\ExtensionTok}[1]{#1}
\newcommand{\FloatTok}[1]{\textcolor[rgb]{0.00,0.00,0.81}{#1}}
\newcommand{\FunctionTok}[1]{\textcolor[rgb]{0.00,0.00,0.00}{#1}}
\newcommand{\ImportTok}[1]{#1}
\newcommand{\InformationTok}[1]{\textcolor[rgb]{0.56,0.35,0.01}{\textbf{\textit{#1}}}}
\newcommand{\KeywordTok}[1]{\textcolor[rgb]{0.13,0.29,0.53}{\textbf{#1}}}
\newcommand{\NormalTok}[1]{#1}
\newcommand{\OperatorTok}[1]{\textcolor[rgb]{0.81,0.36,0.00}{\textbf{#1}}}
\newcommand{\OtherTok}[1]{\textcolor[rgb]{0.56,0.35,0.01}{#1}}
\newcommand{\PreprocessorTok}[1]{\textcolor[rgb]{0.56,0.35,0.01}{\textit{#1}}}
\newcommand{\RegionMarkerTok}[1]{#1}
\newcommand{\SpecialCharTok}[1]{\textcolor[rgb]{0.00,0.00,0.00}{#1}}
\newcommand{\SpecialStringTok}[1]{\textcolor[rgb]{0.31,0.60,0.02}{#1}}
\newcommand{\StringTok}[1]{\textcolor[rgb]{0.31,0.60,0.02}{#1}}
\newcommand{\VariableTok}[1]{\textcolor[rgb]{0.00,0.00,0.00}{#1}}
\newcommand{\VerbatimStringTok}[1]{\textcolor[rgb]{0.31,0.60,0.02}{#1}}
\newcommand{\WarningTok}[1]{\textcolor[rgb]{0.56,0.35,0.01}{\textbf{\textit{#1}}}}
\usepackage{graphicx}
\makeatletter
\def\maxwidth{\ifdim\Gin@nat@width>\linewidth\linewidth\else\Gin@nat@width\fi}
\def\maxheight{\ifdim\Gin@nat@height>\textheight\textheight\else\Gin@nat@height\fi}
\makeatother
% Scale images if necessary, so that they will not overflow the page
% margins by default, and it is still possible to overwrite the defaults
% using explicit options in \includegraphics[width, height, ...]{}
\setkeys{Gin}{width=\maxwidth,height=\maxheight,keepaspectratio}
% Set default figure placement to htbp
\makeatletter
\def\fps@figure{htbp}
\makeatother
\setlength{\emergencystretch}{3em} % prevent overfull lines
\providecommand{\tightlist}{%
  \setlength{\itemsep}{0pt}\setlength{\parskip}{0pt}}
\setcounter{secnumdepth}{5}
\ifLuaTeX
  \usepackage{selnolig}  % disable illegal ligatures
\fi

\begin{document}
\maketitle

\hypertarget{introduction}{%
\section{Introduction}\label{introduction}}

This document is intended to give you ``guidelines and generic steps''
to follow when working on GMACS and release a new version. Its aims are
to show you how to use this Github repository to modify GMACS while
tracking code changes. The idea here is to make this workflow easy to
follow so we will work with Github with no command-line interface using
the graphical interface for Git
\emph{\href{https://desktop.github.com/}{Github Desktop}}.

We assume you already know how to use Rstudio with projects. If you have
never used Github (and/or Githhub Desktop) with Rstudio, you can see the
following online workshops: the
\href{https://happygitwithr.com/}{``Happy Git and Github for the useR''}
from \emph{Jenny Brian} and the
\href{https://rverse-tutorials.github.io/RWorkflow-NWFSC-2021/index.html}{R
Workflow} workshop from \emph{Elizabeth Holmes}.

Throughout this document, we will use a number of \texttt{R} code to
update a new version of GMACS. The functions used are available in the
\href{https://github.com/GMACS-project/gmr}{\texttt{gmr}} package which
will be useful for working with gmacs. This package is available on the
organization's repository. To get it install you will need the
\href{https://www.r-project.org/nosvn/pandoc/devtools.html}{\texttt{devtools}}
package. In the course of this document we will also need different
packages to use AD Model Builder
(\href{https://www.admb-project.org/}{ADMB}) to compile and build the
GMACS executable.

As the repository is private, you will have to set your credentials in R
to be able to download the package. The following code help you to do
this. Please use the address of your Github account (the one that gives
you access to the organization's repository in Github).

\begin{Shaded}
\begin{Highlighting}[]

\NormalTok{.pack }\OtherTok{\textless{}{-}} \StringTok{"gmr"}
\NormalTok{.DirSrc }\OtherTok{\textless{}{-}} \StringTok{"GMACS{-}project/gmr"}
\NormalTok{.Username }\OtherTok{\textless{}{-}} \StringTok{""}                     \CommentTok{\# the name of your Github}
\NormalTok{.UserEmail}\OtherTok{\textless{}{-}} \StringTok{""}        \CommentTok{\# Your email address associated to your Github account}

\CommentTok{\# Remove the package if you already got it installed on your computer}
\FunctionTok{remove.packages}\NormalTok{(.pack, }\AttributeTok{lib=}\StringTok{"\textasciitilde{}/R/win{-}library/4.1"}\NormalTok{)}

\CommentTok{\# Set config}
\NormalTok{usethis}\SpecialCharTok{::}\FunctionTok{use\_git\_config}\NormalTok{(}\AttributeTok{user.name =}\NormalTok{ .Username, }\AttributeTok{user.email =}\NormalTok{ .UserEmail)}

\CommentTok{\# Go to github page to generate token}
\NormalTok{usethis}\SpecialCharTok{::}\FunctionTok{create\_github\_token}\NormalTok{() }

\CommentTok{\# Paste your PAT into pop{-}up that follows...}
\NormalTok{credentials}\SpecialCharTok{::}\FunctionTok{set\_github\_pat}\NormalTok{()}

\CommentTok{\# Now remotes::install\_github() will work}
\NormalTok{devtools}\SpecialCharTok{::}\FunctionTok{install\_github}\NormalTok{(.DirSrc)}
\end{Highlighting}
\end{Shaded}

The following code proceeds the installation of the \texttt{{[}gmr{]}}
package on your machine and load the various libraries.

\begin{Shaded}
\begin{Highlighting}[]
\FunctionTok{rm}\NormalTok{(}\AttributeTok{list=}\FunctionTok{ls}\NormalTok{())     }\CommentTok{\# Clean your R session}

\CommentTok{\# Set the working directory as the directory of this document{-}{-}{-}{-}}
\FunctionTok{setwd}\NormalTok{(}\FunctionTok{dirname}\NormalTok{(rstudioapi}\SpecialCharTok{::}\FunctionTok{getActiveDocumentContext}\NormalTok{()}\SpecialCharTok{$}\NormalTok{path))}
\CommentTok{\# check your directory}
\FunctionTok{getwd}\NormalTok{()}

\CommentTok{\# Get installed devtools to be able to install the gmr package{-}{-}{-}{-}}
\FunctionTok{install.packages}\NormalTok{(}\StringTok{"devtools"}\NormalTok{)                      }\CommentTok{\# install devtools}
\NormalTok{devtools}\SpecialCharTok{::}\FunctionTok{install\_github}\NormalTok{(}\StringTok{"GMACS{-}project/gmr"}\NormalTok{)       }\CommentTok{\# install devtools}

\CommentTok{\# Install and load the packages{-}{-}{-}{-}}

\FunctionTok{library}\NormalTok{(gmr)}
\FunctionTok{install.packages}\NormalTok{(}\StringTok{"miceadds"}\NormalTok{)                      }\CommentTok{\# install miceadds}
\FunctionTok{library}\NormalTok{(miceadds)}
\FunctionTok{install.packages}\NormalTok{(}\StringTok{"PBSadmb"}\NormalTok{)                       }\CommentTok{\# install PBSadmb}
\FunctionTok{library}\NormalTok{(PBSadmb)}
\FunctionTok{install.packages}\NormalTok{(}\StringTok{"knitr"}\NormalTok{)                         }\CommentTok{\# install knitr}
\FunctionTok{library}\NormalTok{(knitr)}
\CommentTok{\# install.packages("svDialogs")                     \# install svDialogs}
\FunctionTok{library}\NormalTok{(svDialogs)}



\CommentTok{\# Source additional functions {-} They will be add to gmr latter}
\FunctionTok{source}\NormalTok{(}\StringTok{"Additional\_functions.R"}\NormalTok{)}
\FunctionTok{source}\NormalTok{(}\StringTok{"GetGmacsExe.R"}\NormalTok{)}
\end{Highlighting}
\end{Shaded}

\texttt{Codes} in this document are summarized in a clean and runnable
\texttt{.R} script
(\href{https://github.com/GMACS-project/GMACS_Assessment_code/blob/main/GMACS/UpdateGMACS.R}{UpdateGMACS.R})
which you should use whenever you want to release a new version of
GMACS. This R script will guide you through all the steps of the
procedure making the GMACS upgrade process easier and more transparent.

\hypertarget{set-up-for-using-this-workflow}{%
\section{Set-up for using this
workflow}\label{set-up-for-using-this-workflow}}

As a member of the
\href{https://github.com/GMACS-project}{GMACS-project} organization, you
probably already have a \href{https://github.com/}{Github} account; if
not, you will need one if you want to work with this repository and
potentially act as an active developer. You will also need to get
installed R (or Rstudio) and Github Desktop on your computer. Below are
the links to install these programs:

\begin{itemize}
\tightlist
\item
  Install \href{https://cran.r-project.org/}{R}
\item
  Install
  \href{https://www.rstudio.com/products/rstudio/download/}{Rstudio}
\item
  Install \href{https://desktop.github.com/}{Github Desktop}
\end{itemize}

If you want to link your R/Rstudio to your Github account so that you
can push your changes from Rstudio directly to Github, please refer to
the two workshops listed above. I will not cover this topic in this
document.

\hypertarget{lets-get-the-gmacs_assessment_code-repository-on-your-computer}{%
\section{\texorpdfstring{Let's get the \emph{GMACS\_Assessment\_code}
repository on your
computer}{Let's get the GMACS\_Assessment\_code repository on your computer}}\label{lets-get-the-gmacs_assessment_code-repository-on-your-computer}}

I am going to show you a workflow to get the
\href{https://github.com/GMACS-project/GMACS_Assessment_code}{GMACS\_Assessment\_code}
repository on your computer. Here, you have two options:

\begin{enumerate}
\def\labelenumi{\arabic{enumi}.}
\tightlist
\item
  You want to contribute to the development of GMACS: you will therefore
  need to \protect\hyperlink{Fork-repo}{\textbf{fork}} the repository to
  be able to directly push up changes to GMACS to the organization's
  repository.
\item
  You only want to \protect\hyperlink{Copy-repo}{\textbf{clone}} the
  repository of the organization and modify it to your liking without
  pushing up modifications to the repository of the organization. In
  this case, you will not contribute directly to the development of
  GMACS.
\end{enumerate}

\hypertarget{Fork-repo}{%
\subsection{\texorpdfstring{Fork the
\href{https://github.com/GMACS-project/GMACS_Assessment_code}{GMACS\_Assessment\_code}
repository}{Fork the GMACS\_Assessment\_code repository}}\label{Fork-repo}}

By \textbf{\emph{forking}} this repository, you will be able to
contribute to the organization's repository. If you just want to copy it
and then modify it for your own purpose on you computer, please follow
the steps described in the \protect\hyperlink{Copy-repo}{Copy a
GMACS-project repository} section.

\textbf{Warning:} \emph{the following workflow will provide you with
``real'' access to the GMACS repository. Therefore, any changes you may
make to this repository can become ``permanent''. Please, never delete
an important file or folder if you are not sure. Thank you.}

\begin{enumerate}
\def\labelenumi{\arabic{enumi}.}
\tightlist
\item
  In Github Desktop, click \emph{File \textgreater{} Clone Repository}
\item
  In the \emph{``Clone a repository''} popup window, chose the URL tab
  and paste the GMACS\_Assessment\_code repository url.
\item
  Check the selected folder in the local path and click \texttt{Clone}.
\end{enumerate}

You are now ready to make some changes to GMACS, commit them and push up
them to the organization repository. You can also create a new RStudio
project in the folder holding the
\href{https://github.com/GMACS-project/GMACS_Assessment_code}{GMACS\_Assessment\_code}
repository on your computer and start to work with GMACS outputs.

\hypertarget{Copy-repo}{%
\subsection{\texorpdfstring{Clone the
\href{https://github.com/GMACS-project/GMACS_Assessment_code}{GMACS\_Assessment\_code}
repository}{Clone the GMACS\_Assessment\_code repository}}\label{Copy-repo}}

Here I show you how to copy the
\href{https://github.com/GMACS-project/GMACS_Assessment_code}{GMACS\_Assessment\_code}
repository on your computer. By following these steps, you will not be
able to contribute to the
\href{https://github.com/GMACS-project}{GMACS-project} organisation but
you will have the possibility to adapt this repository for your purpose
without fear of impacting the organization's repository. Below are the
steps to follow:

\begin{enumerate}
\def\labelenumi{\arabic{enumi}.}
\tightlist
\item
  Get and copy the
  \href{https://github.com/GMACS-project/GMACS_Assessment_code}{url} of
  the
  \href{https://github.com/GMACS-project/GMACS_Assessment_code}{GMACS\_Assessment\_code}
  repository.
\item
  Go to you Github account. It should be github.com/your-name.
\item
  In Github, click the \texttt{+} in top right and select
  \texttt{import\ repository}.
\item
  Paste the url in \textbf{\emph{the Your old repository's clone URL}}
  section and give a name for this new repo. You have now the
  \href{https://github.com/GMACS-project/GMACS_Assessment_code}{GMACS\_Assessment\_code}
  on your own Github (i.e., you have an url looking like
  \emph{github.com/your-name/GMACS\_Assessment\_code} if you kept the
  same name as the one of the organization).
\item
  You can now clone this repository to your computer following the same
  steps as those used to clone the
  \href{https://github.com/GMACS-project/GMACS_Assessment_code}{GMACS\_Assessment\_code}
  repository from the
  \href{https://github.com/GMACS-project}{organization} website.
\end{enumerate}

\hypertarget{lets-work-on-a-new-version-of-gmacs}{%
\section{Let's work on a new version of
GMACS}\label{lets-work-on-a-new-version-of-gmacs}}

In the
\textbf{\href{https://github.com/GMACS-project/GMACS_Assessment_code/tree/main/GMACS}{GMACS}}
folder of the
\href{https://github.com/GMACS-project/GMACS_Assessment_code}{GMACS\_Assessment\_code}
repository, two subfolders should attract your attention when you want
to implement a new version of GMACS (I'm talking about development when
you make a change (no matter how small) to the GMACS code):

\begin{enumerate}
\def\labelenumi{\arabic{enumi}.}
\tightlist
\item
  The
  \textbf{\href{https://github.com/GMACS-project/GMACS_Assessment_code/tree/main/GMACS/Dvpt_Version}{Dvpt\_Version}}
  folder: this folder contains the version of GMACS
  \textbf{\emph{currently in development}}. This means that this version
  is still being tested after some modifications and has not been
  released as a stable version.
\item
  The
  \textbf{\href{https://github.com/GMACS-project/GMACS_Assessment_code/tree/main/GMACS/Latest_Version}{Latest\_Version}}
  folder: this folder holds the latest stable and tested version of
  GMACS.
\end{enumerate}

These two subfolders contain all the hardware you need to run GMACS
either in a ``development way'' or to get a ``stable'' stock assessment.

\textbf{If you want to make new changes to the GMACS code, please work
only in the subfolder
\emph{\href{https://github.com/GMACS-project/GMACS_Assessment_code/tree/main/GMACS/Dvpt_Version}{Dvpt\_Version}}.}

In the following, I give you some guidelines about how correctly update
a GMACS version and push up those change on Github. Obviously, this
assumes that you have previously forked (and not cloned) the
organization's repo on your computer.

Let's take the example of incorporating time-varying natural mortality
into GMACS with a focus on snow crab. I will not go through the
implementation of the code itself but simply give you the path to follow
to:

\begin{enumerate}
\def\labelenumi{\roman{enumi}.}
\tightlist
\item
  keep track of these changes,
\item
  check that they do not impact the obtainment of a new executable
  (i.e., that the code can be compiled),
\item
  analyze the impact of these changes on the assessment for a stock
  (i.e., compare with the last stable version of GMACS the results of
  the assessment using this version under development) and,
\item
  release this new stable version of GMACS to the community.
\end{enumerate}

\emph{Reminder:} \textbf{You will be working in the
\href{https://github.com/GMACS-project/GMACS_Assessment_code/tree/main/GMACS/Dvpt_Version}{Dvpt\_Version}
subfolder.}

\hypertarget{modify-the-gmacsbase.tpl}{%
\subsection{Modify the gmacsbase.tpl}\label{modify-the-gmacsbase.tpl}}

In the
\href{https://github.com/GMACS-project/GMACS_Assessment_code/tree/main/GMACS/Dvpt_Version}{Dvpt\_Version}
subfolder, open the gmacsbase.tpl file in an editor and incorporate the
new functions/variables to get time-varying natural mortality
configuration. This operation does not require any ``new entries'' or
variable declarations in the .ctl, .dat or .prj files so that the change
you make will remain (first) at the gmacsbase.tpl file level.

\hypertarget{check-compilation-and-build-the-executable}{%
\subsection{Check compilation and build the
executable}\label{check-compilation-and-build-the-executable}}

Once you are done with the configuration of time-varying natural
mortality in GMACS, you now need to check that you are still able to
compile the model and build the executable. To build the source files of
GMACS into an executable, you will need to call ADMB. Gmacs can now be
compile, build and run using \texttt{R}. The only requirement is to
provide him with specific directories so it will be able locate ADMB and
a C/C++ compiler.

To provide such directories, you can use the
\href{https://github.com/GMACS-project/GMACS_Assessment_code/blob/main/GMACS/ADpaths.txt}{ADpaths.txt}
file available on Github. You will need to modify the paths accordingly
with your setups on your computer. This text file holds the information
of specific R variable names and the pathway to use them. \textbf{Please
do not modify the R variable names in this file otherwise you will have
some troubles and will not be able to build the GMACS executable.} By
default, this file is locate in the current working directory (i.e., in
the GMACS folder).

Running this procedure, you will be asked to provide a name for the new
version of GMACS. This name will be the one used at the end of all your
modifications for this new version so if you have to realize multiple
compilation, please stay ``consistent'' in the name of the version you
provide.

The following code compiles and build the GMACS executable.

\begin{Shaded}
\begin{Highlighting}[]

\CommentTok{\# Define the name of the file containing the different pathways needed to build}
\CommentTok{\# the GMACS executable }
\NormalTok{.ADMBpaths }\OtherTok{\textless{}{-}} \StringTok{"ADpaths.txt"}

\CommentTok{\# Run the GetGmacsExe function}
\FunctionTok{GetGmacsExe}\NormalTok{()}
\end{Highlighting}
\end{Shaded}

While the command is running, you will see on the console exactly what
is going on. If the compilation worked, you should have a new executable
named \texttt{gmacs.exe} in the root of the
\href{https://github.com/GMACS-project/GMACS_Assessment_code/tree/main/GMACS/Dvpt_Version}{Dvpt\_Version}
directory. You are now ready to test this new version. Let's first
modify (if needed) the .dat, .ctl, .prj files.

\hypertarget{modify-the-.dat-.ctl-.prj-files}{%
\subsection{Modify the .dat, .ctl, .prj
files}\label{modify-the-.dat-.ctl-.prj-files}}

In our case, we only need to modify the control parameters for the Time
varying natural mortality rates. In the .ctl file, we therefore changed
the \emph{type} of model for natural mortality, the \emph{phase of
estimation}, the \emph{standard deviation} of the deviations for the
random walk, the number of \emph{step-changes} as we used blocked
changes, we define the specific years position of the knots and finally
we specified the values for the initial parameters, their low bound,
their high bound and their phase.

\hypertarget{run-the-development-version-using-a-case-study-snow-crab}{%
\subsection{Run the Development version using a case study: Snow
crab}\label{run-the-development-version-using-a-case-study-snow-crab}}

You are now going to run the GMACS development version for a specific
case study. Let's take the snow crab stock as example for this
demonstration. GMACS can be run using the
\href{https://github.com/GMACS-project/GMACS_Assessment_code/blob/main/GMACS/Run_GMACS.Rmd}{Run\_GMACS}
Rmarkdown document which call the \texttt{GMACS()} function.

First define the characteristics of this analysis:

\begin{Shaded}
\begin{Highlighting}[]

\CommentTok{\# Species of interest}
\NormalTok{.Spc }\OtherTok{\textless{}{-}}\FunctionTok{c}\NormalTok{(}
  \StringTok{"SNOW\_M\_time\_varying"}
\NormalTok{  )}

\CommentTok{\# Names of the GMACS version to consider}
\NormalTok{.GMACS\_version }\OtherTok{\textless{}{-}} \FunctionTok{c}\NormalTok{(}
  \StringTok{"Dvpt\_Version"}
\NormalTok{  )}

\CommentTok{\# Define directories}
\NormalTok{.VERSIONDIR }\OtherTok{\textless{}{-}} \FunctionTok{c}\NormalTok{(}
  \FunctionTok{paste0}\NormalTok{(}\FunctionTok{getwd}\NormalTok{(), }\StringTok{"/Dvpt\_Version/"}\NormalTok{)}
\NormalTok{  )}

\CommentTok{\# Need to conpile the model?}
  \CommentTok{\# vector of length(.GMACS\_version)}
  \CommentTok{\# 0: GMACS is not compiled. This assumes that an executable exists in the directory of the concerned version.}
  \CommentTok{\# 1: GMACS is compiles}
\NormalTok{.COMPILE }\OtherTok{\textless{}{-}} \DecValTok{0}       \CommentTok{\# You already compile and build the executable}

\CommentTok{\# Run GMACS}
\NormalTok{.RUN\_GMACS }\OtherTok{\textless{}{-}} \ConstantTok{TRUE}

\CommentTok{\# Use latest available data for the assessment?}
\NormalTok{.LastAssDat }\OtherTok{\textless{}{-}} \ConstantTok{TRUE}

\CommentTok{\# Define the directories for ADMB}
\NormalTok{.ADMBpaths }\OtherTok{\textless{}{-}} \StringTok{"ADpaths.txt"}

\CommentTok{\# Show Rterminal}
\NormalTok{.VERBOSE }\OtherTok{\textless{}{-}} \ConstantTok{TRUE}
\end{Highlighting}
\end{Shaded}

Then, run the model:

\begin{Shaded}
\begin{Highlighting}[]
\NormalTok{res }\OtherTok{\textless{}{-}} \FunctionTok{GMACS}\NormalTok{(}
  \AttributeTok{Spc =}\NormalTok{ .Spc,}
  \AttributeTok{GMACS\_version =}\NormalTok{ .GMACS\_version,}
  \AttributeTok{Dir =}\NormalTok{ .VERSIONDIR,}
  \AttributeTok{compile =}\NormalTok{ .COMPILE,}
  \AttributeTok{run =}\NormalTok{ .RUN\_GMACS,}
  \AttributeTok{LastAssDat =}\NormalTok{ .LastAssDat,}
  \AttributeTok{ADMBpaths =}\NormalTok{ .ADMBpaths,}
\NormalTok{  verbose }\OtherTok{\textless{}{-}}\NormalTok{ .VERBOSE}
\NormalTok{)}
\end{Highlighting}
\end{Shaded}

All the output files resulting from this run are stored in the
*GMACS\Dvpt\_Version\build\SNOW\_M\_time\_varying* directory. You are
now ready to compare the outputs of this GMACS development version with
the ones coming from the latest assessment.

\hypertarget{compare-the-results-of-this-new-version-with-the-results-from-the-last-assessment}{%
\subsection{Compare the results of this new version with the results
from the last
assessment}\label{compare-the-results-of-this-new-version-with-the-results-from-the-last-assessment}}

In the
\href{https://github.com/GMACS-project/GMACS_Assessment_code}{GMACS\_Assessment\_code}
repository, the required input files and specific outputs files from the
latest assessment for all stocks are are stored in the folder
\href{https://github.com/GMACS-project/GMACS_Assessment_code/tree/main/Assessments}{Assessments}.
For each stocks, this folder contains the models that have been tested,
presented and selected by the Crab Plan Team and the Scientific and
Statistical Committee to realize the assessment of each crab stock.

This comparison can be done using the
\href{https://github.com/GMACS-project/GMACS_Assessment_code/blob/main/GMACS/Compare_Version_VS_Last_Assessment.Rmd}{Compare\_Version\_VS\_Last\_Assessment}
Rmarkdown document. Here, we will compare the results of the stock
assessment for snow crab between the latest assessment and the
development version of GMACS. In the same way as previously, the
comparison is done by calling the the \texttt{GMACS()} function. You
should have been able to do the run and the comparison at the same time
i.e., calling the \texttt{GMACS()} function only once but for the
purpose of this document we split the steps. Obviously, we are not going
again through the entire run here. You have first to modify the
\texttt{.GMACS\_version} and the \texttt{.VERSIONDIR} variables to
consider the last assessment and turn off the \texttt{.RUN\_GMACS}
variable to avoid going again through the run. You will also need to
specify the \texttt{.ASS} and \texttt{.ASSMOD\_NAMES} variables that
indicate you want to consider the last assessment in the comparison and
the name of the model used for this assessment and of course, you will
have to turn on the \texttt{.MAKE\_Comp} variable to inticate that you
want to make comparison between the two versions.

Changes the parameters for the comparison:

\begin{Shaded}
\begin{Highlighting}[]

\CommentTok{\# Names of the GMACS version to consider for run}
\NormalTok{.GMACS\_version }\OtherTok{\textless{}{-}} \FunctionTok{c}\NormalTok{(}
  \StringTok{"Last\_Assessment"}\NormalTok{,}
  \StringTok{"Latest\_Version"}
\NormalTok{  )}

\CommentTok{\# Define directory}
\NormalTok{.VERSIONDIR }\OtherTok{\textless{}{-}} \FunctionTok{c}\NormalTok{(}
  \FunctionTok{paste0}\NormalTok{(}\FunctionTok{dirname}\NormalTok{(}\FunctionTok{getwd}\NormalTok{()), }\StringTok{"/Assessments/"}\NormalTok{),}
  \FunctionTok{paste0}\NormalTok{(}\FunctionTok{getwd}\NormalTok{(), }\StringTok{"/Latest\_Version/"}\NormalTok{)}
\NormalTok{)}

\CommentTok{\# Run GMACS}
\NormalTok{.RUN\_GMACS }\OtherTok{\textless{}{-}} \ConstantTok{FALSE}

\CommentTok{\# Use Last Assessment for comparison?}
\CommentTok{\# If yes, you must provide the names of the model for each species in the variable .ASSMOD\_NAMES}
\CommentTok{\# Those model folder must have to be hold in the folder Assessments}
\NormalTok{.ASS }\OtherTok{\textless{}{-}} \ConstantTok{TRUE}

\CommentTok{\# names of the model for the last assessment {-} Only useful if comparison is made.}
\CommentTok{\# if all stocks are considered they have to be ordered as follow:}
\CommentTok{\# "AIGKC/EAG" / "AIGKC/WAG" / "BBRKC" / "SMBKC" / "SNOW"}
\NormalTok{.ASSMOD\_NAMES }\OtherTok{\textless{}{-}} \StringTok{"model\_21\_g"}

\CommentTok{\# Do comparison?}
\NormalTok{.MAKE\_Comp }\OtherTok{\textless{}{-}} \ConstantTok{TRUE}
\end{Highlighting}
\end{Shaded}

Call the GMACS() function:

\begin{Shaded}
\begin{Highlighting}[]

\NormalTok{tables }\OtherTok{\textless{}{-}} \FunctionTok{GMACS}\NormalTok{(}\AttributeTok{Spc =}\NormalTok{ .Spc, }\AttributeTok{GMACS\_version =}\NormalTok{ .GMACS\_version,}
        \AttributeTok{Dir =}\NormalTok{ .VERSIONDIR,}
        \AttributeTok{ASS =}\NormalTok{ .ASS,}
        \AttributeTok{AssMod\_names =}\NormalTok{ .ASSMOD\_NAMES,}
        \AttributeTok{run =}\NormalTok{ .RUN\_GMACS,}
        \AttributeTok{make.comp =}\NormalTok{ .MAKE\_Comp)}
\CommentTok{\# print(tables)}
\end{Highlighting}
\end{Shaded}

If you are satisfied with the results of the comparison between these
two versions of GMACS, you are now ready to formalize and release this
new version and spread it to the community.

\hypertarget{spread-the-new-gmacs-version-to-the-whole-community}{%
\section{Spread the new GMACS version to the whole
community}\label{spread-the-new-gmacs-version-to-the-whole-community}}

The first step before releasing the new version is to update the
\href{https://github.com/GMACS-project/GMACS_Assessment_code/tree/main/GMACS/Latest_Version}{Latest\_Version}
folder with the new code of GMACS.

\hypertarget{copy-the-latest-version-to-the-latest_version-folder}{%
\subsection{Copy the latest version to the Latest\_Version
folder}\label{copy-the-latest-version-to-the-latest_version-folder}}

Luckily and for the sake of efficiency and transparency, you don't have
to do anything by hand. The \texttt{UpdateGMACS()} function allows you
to:

\begin{enumerate}
\def\labelenumi{\roman{enumi})}
\tightlist
\item
  Copy and paste all the files you used for the GMACS development
  version to the
  \href{https://github.com/GMACS-project/GMACS_Assessment_code/tree/main/GMACS/Latest_Version}{Latest\_Version}
  folder
\item
  Compile this new release version in the
  \href{https://github.com/GMACS-project/GMACS_Assessment_code/tree/main/GMACS/Latest_Version}{Latest\_Version}
  folder and get everything ready to use it
\end{enumerate}

\hypertarget{push-up-changes-to-the-organizations-repository}{%
\subsection{Push up changes to the organization's
repository}\label{push-up-changes-to-the-organizations-repository}}

\hypertarget{re-run-the-assessments-with-the-latest-version-optional}{%
\subsection{Re-run the assessments with the latest version
(optional)}\label{re-run-the-assessments-with-the-latest-version-optional}}

\end{document}
